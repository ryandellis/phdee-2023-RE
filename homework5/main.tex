\documentclass{article}
\usepackage[utf8]{inputenc}
\usepackage{hyperref}
\usepackage[letterpaper, portrait, margin=1in]{geometry}
\usepackage{enumitem}
\usepackage{amsmath}
\usepackage{booktabs}
\usepackage{graphicx}

\usepackage{hyperref}
\hypersetup{
colorlinks=true,
    linkcolor=black,
    filecolor=black,      
    urlcolor=blue,
    citecolor=black,
}
\usepackage{natbib}

\usepackage{titlesec}
  
\title{Homework 5}
\author{Ryan Ellis}
\date{Spring semester 2023}
  
\begin{document}
  
\maketitle

\section{Python}
\subsection{}
\begin{itemize}
    \item The coefficient on the endogenous $mpg$ is -22.21. A unit increase in $mpg$ corresponds to a decrease in price of \$22.21.
\end{itemize}

\subsection{}
\begin{itemize}
    \item We should be concerned primarily with omitted variable bias, or equivalently, the possibility that $mpg$ is a confounder correlated with both $price$ and the error term. It's unlikely that we have measurement error or simultaneity in this particular application.
\end{itemize}

\subsection{}
\begin{table}[!htbp] \centering
\begin{tabular}{@{\extracolsep{5pt}}lccc}
\\[-1.8ex]\hline
\hline \\[-1.8ex]
& \multicolumn{3}{c}{\textit{Dependent variable: $price$}} \
\cr \cline{3-4}
\\[-1.8ex] & (1) & (2) & (3) \\
\hline \\[-1.8ex]
 car & -4676.09$^{***}$ & -4732.67$^{***}$ & -90156.39$^{}$ \\
  & (574.37) & (573.29) & (226687.35) \\
 const & 17627.64$^{***}$ & 17441.23$^{***}$ & -264024.20$^{}$ \\
  & (1754.87) & (1751.12) & (746919.27) \\
 $\hat{mpg(a)}$ & 150.43$^{**}$ & & \\
  & (62.16) & & \\
 $\hat{mpg(b)}$ & & 157.06$^{**}$ & \\
  & & (62.02) & \\
 $\hat{mpg(c)}$ & & & 10165.74$^{}$ \\
  & & & (26559.83) \\
\hline \\[-1.8ex]
 First-stage F & [[75.4640828]] & [[75.76900674]] & [[0.0003864]] \\
 Observations & 1,000 & 1,000 & 1,000 \\
 $R^2$ & 0.20 & 0.20 & 0.19 \\
 Adjusted $R^2$ & 0.19 & 0.19 & 0.19 \\
 Residual Std. Error & 3481.08 & 3480.12 & 3491.04  \\
 F Statistic & 121.62$^{***}$  & 121.97$^{***}$  & 118.09$^{***}$  \\
\hline
\hline \\[-1.8ex]
\textit{Note:} & \multicolumn{3}{r}{$^{*}$p$<$0.1; $^{**}$p$<$0.05; $^{***}$p$<$0.01} \\
\end{tabular}
\end{table}

\clearpage

\subsection{}
\begin{itemize}
    \item Using GMM, the second-stage coefficient of interest ($mpg$) is 150.43, with S.E. 63.05. The point estimate is identical to treatment (1) in the table above, but with slightly larger errors, likely due to a suboptimal weighting matrix in the GMM estimation. GMM is more efficient than 2SLS when there are multiple instruments. In this case, there is only one.
\end{itemize}


\section{Stata}

\subsection{}


\end{document}